% Verwendet: Beamer-Class
% https://en.wikibooks.org/wiki/LaTeX/Presentations
% http://mirrors.ibiblio.org/CTAN/macros/latex/contrib/beamer/doc/beameruserguide.pdf
% 
% Sehr wichtig: jede Folie ist in \begin{frame} ... \end{frame} eingeschlossen.
% lstlisting(also Source Code) kann nur in Folien verwendet werden, wenn ihr als
% Parameter \begin{frame}[fragile] angebt! Ansonsten kompiliert das Dokument nicht, und ihr habt keine Ahnung warum :/

\PassOptionsToPackage{table}{xcolor}
\documentclass[mathserif]{beamer}
\usetheme{CambridgeUS}

\useinnertheme{circles}
\usepackage[ngerman]{babel}
\usepackage[utf8]{inputenc}
\usepackage{listings}
\usepackage{tabularx}
\usepackage{graphicx}
\definecolor{gray1}{gray}{0.90}
\definecolor{gray2}{gray}{0.95}
\definecolor{myred}{rgb}{0.64, 0, 0}
\setbeamertemplate{navigation symbols}{} %Lässt hässliche Navigationssysmbole verschwinden
\title{Implementierung des Sinh}
\subtitle{in x86-Assembly}
\author
{Kevin Holm, Deniz Candas, Jakob Mezger}
\date{09.08.2017}
\lstdefinelanguage
[x64]{Assembler}     % add a "x64" dialect of Assembler
[x86masm]{Assembler} % based on the "x86masm" dialect
% with these extra keywords:
{morekeywords={CDQE,CQO,CMPSQ,CMPXCHG16B,JRCXZ,LODSQ,MOVSXD, %
		POPFQ,PUSHFQ,SCASQ,STOSQ,IRETQ,RDTSCP,SWAPGS, %
		rax,rdx,rcx,rbx,rsi,rdi,rsp,rbp, %
		r8,r8d,r8w,r8b,r9,r9d,r9w,r9b, %
		r10,r10d,r10w,r10b,r11,r11d,r11w,r11b, %
		r12,r12d,r12w,r12b,r13,r13d,r13w,r13b, %
		r14,r14d,r14w,r14b,r15,r15d,r15w,r15b}} % etc.
\lstdefinestyle{myPseudo}{language=Java, basicstyle=\tiny, numbers=left, keywordstyle=\color{myred}, frame=tb, commentstyle=\color{blue!70}, xleftmargin = 0.1\textwidth, xrightmargin = 0.1\textwidth}
\lstdefinestyle{myAssembly}{language=[x64]Assembler, basicstyle=\tiny, numbers=left, keywordstyle=\color{myred}, frame=tb, commentstyle=\color{gray}, xleftmargin = 0.1\textwidth, xrightmargin = 0.1\textwidth}


\begin{document}
	
	\frame{\titlepage}
	\begin{frame}
		\tableofcontents
	\end{frame}
	\begin{frame}
		\section{Aufgabe}
		\frametitle{Aufgabe}
		\begin{itemize}
			\item Implementierung des Sinus Hyperbolicus
			%Sinh ist eine Hyperbel-Funktion
			\item Erlaubte Befehle
			\begin{itemize}
				\item x87 FPU-Befehle für Grundrechenarten, Negation
				\item Speicherverwaltungsbefehle
			\end{itemize}
			%Offensichtlich müssen wir uns an die nasm-Syntax und die Linux 64-Bit-ABI halten  
			%Ein-/Ausgabe: IEEE 754 Double Precision 64-Bit
			\item C-Rahmenprogramm
			\begin{itemize}
				\item Validierung
				\item Leistungsmessung 
				\item Vergleich mit Funktion der Standardbibliothek
			\end{itemize}
		\end{itemize}
	\end{frame}
	\begin{frame}
		\section{Verwendete Umgebung}
		\frametitle{Verwendete Umgebung}
		\begin{itemize}
			\item Betriebssystem: Linux Ubuntu 64-Bit LTS 16.04
			\item Assembly-Syntax: nasm
		\end{itemize}
	\end{frame}
	\begin{frame}
		\section{Verwendeter Ansatz}
		\frametitle{Verwendeter Ansatz}
		\framesubtitle{Und dessen Vor- und Nachteile}
		
		\begin{itemize}
			\item Reihenentwicklung: \begin{math}sinh(x) = \sum\limits_{n=0}^{\infty} \frac{x^{2n+1}}{(2n+1)!} \end{math}
			\item Gesteigerte Effizienz durch Rekursion: $term_{0} = x, \ term_{n+1}(x) = term_{n}(x) \times \frac{x}{n+1} \times \frac{x}{n+2}$
		\end{itemize}
	% Beliebige Genauigkeit im Gegensatz zu Berechnung mit Lookup-Table
	% Speicherbedarf unproblematisch, Geschwindigkeit schwer zu beurteilen, Maximale Abweichung von 5% definitiv erfüllt -> ausschlaggebend: einfache Implementierung
	\begin{center}
		\rowcolors{1}{gray1}{gray2}
		\begin{tabularx}{\textwidth}{*{2}{|X}|}
			\hline
			\textbf{Vorteile} & \textbf{Nachteile} \\
			\hline 
			\hline
			Beliebige Genauigkeit & Viele Schleifendurchläufe für \newline genaue Werte \\
			\hline
			Rekursion $\rightarrow$ Einfache \newline Implementierung  & Rekursion $\rightarrow$ Große Stack-\newline Auslastung \\
			\hline
			Wenige Speicherzugriffe & Rechenintensiv \\
			\hline
		\end{tabularx}
	\end{center}
	\end{frame}
	\begin{frame}[fragile]
		\section{Code Review}
		\frametitle{Code Review}
		\framesubtitle{Pseudo Code}
		\subsection{Pseudo Code}
%Auf Abbruchbedingung ("if result is not precise enough") eingehen: (45+5|x|)/8, dabei höchstens 489(für Werte > 710 ||  < -710) mal durchlaufen
\begin{lstlisting}[style=myPseudo, title=Sinh in Pseudocode]
sinh(double x):
double i = 1
double result = x
double term = x

loop:
i = i + 1;
term = term / i
term = term * x

i = i + 1;
term = term / i
term = term * x

result = result + term;
if result is not precise enough jmp loop

return result
\end{lstlisting}
$term_{0} = x, \ term_{n+1}(x) = term_{n}(x) \times \frac{x}{n+1} \times \frac{x}{n+2}$
\end{frame}
\begin{frame}[fragile]
\frametitle{Code Review}
\framesubtitle{Markante Stellen im Code}
\subsection{Markante Stellen im Code}

\begin{lstlisting}[style=myAssembly, title=sinh.asm vor Aufruf der Hauptschleife, firstnumber = 22]
fld st0				;st0 = x, st1 = x

fabs				;st0 = |x|

fdiv st1, st0			;st1 = x/|x|, either 1 or -1
\end{lstlisting}
$\forall x \in \mathbb{R}: sinh(-x) = -sinh(x)$. Also kann man den sinh nur für positive Werte berechnen...
%TODO Kevin fragen, wieso 2 mal fincstp
\begin{lstlisting}[style=myAssembly, title=sinh.asm nach Verlassen der Hauptschleife, firstnumber = 82]
fincstp
fincstp				;st6= previous, st7 = i, st0 = result, 
				;st1 = 1, st2 = x, st3 = sign
fmul st0, st3			;st0 = result * sign
\end{lstlisting}
...und nach der Berechnung das Vorzeichen anpassen.
\end{frame}
\end{document}